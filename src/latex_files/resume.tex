\documentclass[10pt, letterpaper]{article}

% Packages:
\usepackage[
    ignoreheadfoot, % set margins without considering header and footer
    top=2 cm, % seperation between body and page edge from the top
    bottom=2 cm, % seperation between body and page edge from the bottom
    left=2 cm, % seperation between body and page edge from the left
    right=2 cm, % seperation between body and page edge from the right
    footskip=1.0 cm, % seperation between body and footer
    % showframe % for debugging 
]{geometry} % for adjusting page geometry
\usepackage{titlesec} % for customizing section titles
\usepackage{tabularx} % for making tables with fixed width columns
\usepackage{array} % tabularx requires this
\usepackage[dvipsnames]{xcolor} % for coloring text
\definecolor{primaryColor}{RGB}{0, 0, 0} % define primary color
\usepackage{enumitem} % for customizing lists
\usepackage{fontawesome5} % for using icons
\usepackage{amsmath} % for math
\usepackage[
    pdftitle={Zach Norman Resume},
    pdfauthor={Zach},
    pdfcreator={LaTeX with RenderCV},
    colorlinks=true,
    urlcolor=primaryColor
]{hyperref} % for links, metadata and bookmarks
\usepackage[pscoord]{eso-pic} % for floating text on the page
\usepackage{calc} % for calculating lengths
\usepackage{bookmark} % for bookmarks
\usepackage{lastpage} % for getting the total number of pages
\usepackage{changepage} % for one column entries (adjustwidth environment)
\usepackage{paracol} % for two and three column entries
\usepackage{ifthen} % for conditional statements
\usepackage{needspace} % for avoiding page brake right after the section title
\usepackage{iftex} % check if engine is pdflatex, xetex or luatex

% Ensure that generate pdf is machine readable/ATS parsable:
\ifPDFTeX
    \input{glyphtounicode}
    \pdfgentounicode=1
    \usepackage[T1]{fontenc}
    \usepackage[utf8]{inputenc}
    \usepackage{lmodern}
\fi

\usepackage{charter}

% Some settings:
\raggedright
\AtBeginEnvironment{adjustwidth}{\partopsep0pt} % remove space before adjustwidth environment
\pagestyle{empty} % no header or footer
\setcounter{secnumdepth}{0} % no section numbering
\setlength{\parindent}{0pt} % no indentation
\setlength{\topskip}{0pt} % no top skip
\setlength{\columnsep}{0.15cm} % set column seperation
\pagenumbering{gobble} % no page numbering

\titleformat{\section}{\needspace{4\baselineskip}\bfseries\large}{}{0pt}{}[\vspace{1pt}\titlerule]

\titlespacing{\section}{
    % left space:
    -1pt
}{
    % top space:
    0.3 cm
}{
    % bottom space:
    0.2 cm
} % section title spacing

\renewcommand\labelitemi{$\vcenter{\hbox{\small$\bullet$}}$} % custom bullet points
\newenvironment{highlights}{
    \begin{itemize}[
        topsep=0.10 cm,
        parsep=0.10 cm,
        partopsep=0pt,
        itemsep=0pt,
        leftmargin=0 cm + 10pt
    ]
}{
    \end{itemize}
} % new environment for highlights


\newenvironment{highlightsforbulletentries}{
    \begin{itemize}[
        topsep=0.10 cm,
        parsep=0.10 cm,
        partopsep=0pt,
        itemsep=0pt,
        leftmargin=10pt
    ]
}{
    \end{itemize}
} % new environment for highlights for bullet entries

\newenvironment{onecolentry}{
    \begin{adjustwidth}{
        0 cm + 0.00001 cm
    }{
        0 cm + 0.00001 cm
    }
}{
    \end{adjustwidth}
} % new environment for one column entries

\newenvironment{twocolentry}[2][]{
    \onecolentry
    \def\secondColumn{#2}
    \setcolumnwidth{\fill, 4.5 cm}
    \begin{paracol}{2}
}{
    \switchcolumn \raggedleft \secondColumn
    \end{paracol}
    \endonecolentry
} % new environment for two column entries

\newenvironment{threecolentry}[3][]{
    \onecolentry
    \def\thirdColumn{#3}
    \setcolumnwidth{, \fill, 4.5 cm}
    \begin{paracol}{3}
    {\raggedright #2} \switchcolumn
}{
    \switchcolumn \raggedleft \thirdColumn
    \end{paracol}
    \endonecolentry
} % new environment for three column entries

\newenvironment{header}{
    \setlength{\topsep}{0pt}\par\kern\topsep\centering\linespread{1.5}
}{
    \par\kern\topsep
} % new environment for the header

\newcommand{\placelastupdatedtext}{% \placetextbox{<horizontal pos>}{<vertical pos>}{<stuff>}
  \AddToShipoutPictureFG*{% Add <stuff> to current page foreground
    \put(
        \LenToUnit{\paperwidth-2 cm-0 cm+0.05cm},
        \LenToUnit{\paperheight-1.0 cm}
    ){\vtop{{\null}\makebox[0pt][c]{
        \small\color{gray}\textit{Last updated in September 2024}\hspace{\widthof{Last updated in September 2024}}
    }}}%
  }%
}%

% save the original href command in a new command:
\let\hrefWithoutArrow\href

% new command for external links:


\begin{document}
    \newcommand{\AND}{\unskip
        \cleaders\copy\ANDbox\hskip\wd\ANDbox
        \ignorespaces
    }
    \newsavebox\ANDbox
    \sbox\ANDbox{$|$}

    \begin{header}
        \fontsize{20 pt}{20 pt}\selectfont Zach Norman

        \vspace{5 pt}

        \normalsize
        \mbox{Malvern, PA}%
        \kern 5.0 pt%
        \AND%
        \kern 5.0 pt%
        \mbox{\hrefWithoutArrow{mailto:zachnorman02@gmail.com}{zachnorman02@gmail.com}}%
        \kern 5.0 pt%
        \AND%
        % \kern 5.0 pt%
        % \mbox{\hrefWithoutArrow{tel:+1-410-596-6777}{(410) 596-6777}}%
        % \kern 5.0 pt%
        % \AND%
        % \kern 5.0 pt%
        % \mbox{\hrefWithoutArrow{https://yourwebsite.com/}{yourwebsite.com}}%
        % \kern 5.0 pt%
        % \AND%
        \kern 5.0 pt%
        \mbox{\hrefWithoutArrow{https://linkedin.com/in/zachnorman02}{linkedin.com/in/zachnorman02}}%
        \kern 5.0 pt%
        \AND%
        \kern 5.0 pt%
        \mbox{\hrefWithoutArrow{https://github.com/zachnorman02}{github.com/zachnorman02}}%
    \end{header}

    \vspace{5 pt - 0.3 cm}

    \section{Experience}
    \begin{twocolentry}{
        May 2023 - Present
    }
    \textbf{Application Engineer}, Vanguard -- Malvern, PA \end{twocolentry}
    \textit{Python, AWS (Lambda, Glue, S3, IAM, Cloudformation), SQL}
    \vspace{0.10 cm}
    \begin{onecolentry}
      \begin{highlights}
        \item Develop AWS Lambda functions to ingest data from internal data lakes, APIs, and third-party platforms, performing transformations and loading data into S3 buckets, internal databases, or external platforms.
        \item Collaborate with business clients on building and enhancing data pipelines by gathering requirements, translating them into technical specifications, and creating detailed documentation for future reference.
        \item Design data visualizations to monitor and track the success rates of daily business-critical reports to provide insights into common failures and areas for improvement or future development.
      \end{highlights}
    \end{onecolentry}
    \vspace{0.2 cm}
    \begin{twocolentry}{
        Feb 2022 - Nov 2022
    }
    \textbf{Contracted Developer}, Orita \end{twocolentry}
    \textit{Python, Django, HTML, JavaScript, Svelte}
    \vspace{0.10 cm}
    \begin{onecolentry}
      \begin{highlights}
        \item Developed a Python program using Django to automate the generation of business analytics reports from JSON data, leveraging CSS and templates for cohesive and professional report design.
        \item Implemented front-end design for user login, data upload, and e-commerce account connection for companies to easily access and upload their data for analysis.
      \end{highlights}
    \end{onecolentry}
    \vspace{0.2 cm}
    \begin{twocolentry}{
        Jan 2022 - Sep 2022
    }
    \textbf{Developer Co-op}, Baltimore Orioles -- Baltimore, MD \end{twocolentry}
    \textit{Python, JavaScript, Django, React}
    \vspace{0.10 cm}
    \begin{onecolentry}
      \begin{highlights}
        \item Developed data retrieval, aggregation, and user interface for an amateur baseball player dashboard page that allowed for viewing and filtering data on draft-eligible players.
        \item Added edit features to game report audit pages, including the ability to add, delete, or reorder rows of data, and clear data columns, improving ability for coaches and analysts to correct and use game data.
        \item Enchanced multiple AWS Lambda functions responsible for data retrieval, calculation, and upsert to ensure storage of accurate game and player data.
      \end{highlights}
    \end{onecolentry}

    \section{Education}
    \begin{twocolentry}{
      Aug 2023 - Aug 2025
    }
    \textbf{Villanova University}, MS in Applied Statistics and Data Science \end{twocolentry}
    % \vspace{0.10 cm}
    \begin{onecolentry}
        \textit{Villanova, PA}
    \end{onecolentry}
    \vspace{0.2 cm}
    \begin{twocolentry}{
        Sep 2020 - Apr 2023
    }
    \textbf{Northeastern University}, BS in Computer Science and Mathematics\end{twocolentry}
    % \vspace{0.10 cm}
    \begin{onecolentry}
        \textit{Boston, MA}
    \end{onecolentry}
    
    \section{Projects}
        \begin{twocolentry}{
            Jan - Apr 2023
        }
        \textbf{Generate Website} \href{https://github.com/GenerateNU/website}{\textit{GitHub}}\end{twocolentry}
        \vspace{0.10 cm}
        \begin{onecolentry}
            \begin{highlights}
                \item Improved the React-based website for Generate, a student-led product development organization, by implementing mobile responsiveness and resolving styling issues to enhance user experience.
            \end{highlights}
        \end{onecolentry}
        \vspace{0.2 cm}
        \begin{twocolentry}{
            Feb - Apr 2023
        }
        \textbf{NBA Betting Visualization} \href{https://github.com/GenerateNU/website}{\textit{GitHub}}, \href{https://ds4200-s23-class.github.io/project-zach-george-filip-travis/}{\textit{Website}}\end{twocolentry}
        \vspace{0.10 cm}
        \begin{onecolentry}
            \begin{highlights}
                \item Built interactive D3 visualizations to compare betting odds across multiple sportsbooks for NBA games, leveraging data from The Odds API.
            \end{highlights}
        \end{onecolentry}
        \vspace{0.2 cm}
        \begin{twocolentry}{
            Nov - Dec 2022
        }
        \textbf{Image Processor} \href{https://github.com/zachnorman02/webgl-image-processor}{\textit{GitHub}}, \href{https://zachnorman02.github.io/webgl-image-processor/}{\textit{Website}}\end{twocolentry}
        \vspace{0.10 cm}
        \begin{onecolentry}
            \begin{highlights}
                \item  Developed a WebGL-based application that allows users to upload images and apply custom kernel-based filters, enabling real-time image processing in the browser.
            \end{highlights}
        \end{onecolentry}
        \vspace{0.2 cm}

    \section{Technologies}
    \begin{onecolentry}
      \textbf{Languages:} Python, R, JavaScript/TypeScript, Java, SQL, HTML
    \end{onecolentry}
    \begin{onecolentry}
      \textbf{Frameworks/Libraries:} React, Svelte, Django, pandas, Plotly, scikit-learn, D3, tidyverse
    \end{onecolentry}
    \begin{onecolentry}
      \textbf{Tools:} Git, JetBrains IDEs, Visual Studio Code, Jupyter Notebook, AWS (Glue, Lambda, S3, IAM), Figma
    \end{onecolentry}
\end{document}