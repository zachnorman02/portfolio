\documentclass[10pt, letterpaper]{article}

% Packages:
\usepackage[
    ignoreheadfoot, % set margins without considering header and footer
    top=2 cm, % seperation between body and page edge from the top
    bottom=2 cm, % seperation between body and page edge from the bottom
    left=2 cm, % seperation between body and page edge from the left
    right=2 cm, % seperation between body and page edge from the right
    footskip=1.0 cm, % seperation between body and footer
    % showframe % for debugging 
]{geometry} % for adjusting page geometry
\usepackage{titlesec} % for customizing section titles
\usepackage{tabularx} % for making tables with fixed width columns
\usepackage{array} % tabularx requires this
\usepackage[dvipsnames]{xcolor} % for coloring text
\definecolor{primaryColor}{RGB}{0, 0, 0} % define primary color
\usepackage{enumitem} % for customizing lists
\usepackage{fontawesome5} % for using icons
\usepackage{amsmath} % for math
\usepackage[
    pdftitle={Zach Norman Resume},
    pdfauthor={Zach},
    pdfcreator={LaTeX with RenderCV},
    colorlinks=true,
    urlcolor=primaryColor
]{hyperref} % for links, metadata and bookmarks
\usepackage[pscoord]{eso-pic} % for floating text on the page
\usepackage{calc} % for calculating lengths
\usepackage{bookmark} % for bookmarks
\usepackage{lastpage} % for getting the total number of pages
\usepackage{changepage} % for one column entries (adjustwidth environment)
\usepackage{paracol} % for two and three column entries
\usepackage{ifthen} % for conditional statements
\usepackage{needspace} % for avoiding page brake right after the section title
\usepackage{iftex} % check if engine is pdflatex, xetex or luatex

% Ensure that generate pdf is machine readable/ATS parsable:
\ifPDFTeX
    \input{glyphtounicode}
    \pdfgentounicode=1
    \usepackage[T1]{fontenc}
    \usepackage[utf8]{inputenc}
    \usepackage{lmodern}
\fi

\usepackage{charter}

% Some settings:
\raggedright
\AtBeginEnvironment{adjustwidth}{\partopsep0pt} % remove space before adjustwidth environment
\pagestyle{empty} % no header or footer
\setcounter{secnumdepth}{0} % no section numbering
\setlength{\parindent}{0pt} % no indentation
\setlength{\topskip}{0pt} % no top skip
\setlength{\columnsep}{0.15cm} % set column seperation
\pagenumbering{gobble} % no page numbering

\titleformat{\section}{\needspace{4\baselineskip}\bfseries\large}{}{0pt}{}[\vspace{1pt}\titlerule]

\titlespacing{\section}{
    % left space:
    -1pt
}{
    % top space:
    0.3 cm
}{
    % bottom space:
    0.2 cm
} % section title spacing

\renewcommand\labelitemi{$\vcenter{\hbox{\small$\bullet$}}$} % custom bullet points
\newenvironment{highlights}{
    \begin{itemize}[
        topsep=0.10 cm,
        parsep=0.10 cm,
        partopsep=0pt,
        itemsep=0pt,
        leftmargin=0 cm + 10pt
    ]
}{
    \end{itemize}
} % new environment for highlights


\newenvironment{highlightsforbulletentries}{
    \begin{itemize}[
        topsep=0.10 cm,
        parsep=0.10 cm,
        partopsep=0pt,
        itemsep=0pt,
        leftmargin=10pt
    ]
}{
    \end{itemize}
} % new environment for highlights for bullet entries

\newenvironment{onecolentry}{
    \begin{adjustwidth}{
        0 cm + 0.00001 cm
    }{
        0 cm + 0.00001 cm
    }
}{
    \end{adjustwidth}
} % new environment for one column entries

\newenvironment{twocolentry}[2][]{
    \onecolentry
    \def\secondColumn{#2}
    \setcolumnwidth{\fill, 4.5 cm}
    \begin{paracol}{2}
}{
    \switchcolumn \raggedleft \secondColumn
    \end{paracol}
    \endonecolentry
} % new environment for two column entries

\newenvironment{threecolentry}[3][]{
    \onecolentry
    \def\thirdColumn{#3}
    \setcolumnwidth{, \fill, 4.5 cm}
    \begin{paracol}{3}
    {\raggedright #2} \switchcolumn
}{
    \switchcolumn \raggedleft \thirdColumn
    \end{paracol}
    \endonecolentry
} % new environment for three column entries

\newenvironment{header}{
    \setlength{\topsep}{0pt}\par\kern\topsep\centering\linespread{1.5}
}{
    \par\kern\topsep
} % new environment for the header

\newcommand{\placelastupdatedtext}{% \placetextbox{<horizontal pos>}{<vertical pos>}{<stuff>}
  \AddToShipoutPictureFG*{% Add <stuff> to current page foreground
    \put(
        \LenToUnit{\paperwidth-2 cm-0 cm+0.05cm},
        \LenToUnit{\paperheight-1.0 cm}
    ){\vtop{{\null}\makebox[0pt][c]{
        \small\color{gray}\textit{Last updated in September 2024}\hspace{\widthof{Last updated in September 2024}}
    }}}%
  }%
}%

% save the original href command in a new command:
\let\hrefWithoutArrow\href

% new command for external links:


\begin{document}
    \newcommand{\AND}{\unskip
        \cleaders\copy\ANDbox\hskip\wd\ANDbox
        \ignorespaces
    }
    \newsavebox\ANDbox
    \sbox\ANDbox{$|$}

    \begin{header}
        \fontsize{25 pt}{25 pt}\selectfont Zach Norman

        \vspace{5 pt}

        \normalsize
        \mbox{Malvern, PA}%
        \kern 5.0 pt%
        \AND%
        \kern 5.0 pt%
        \mbox{\hrefWithoutArrow{mailto:zachnorman02@gmail.com}{zachnorman02@gmail.com}}%
        \kern 5.0 pt%
        \AND%
        % \kern 5.0 pt%
        % \mbox{\hrefWithoutArrow{tel:+1-410-596-6777}{(410) 596-6777}}%
        % \kern 5.0 pt%
        % \AND%
        % \kern 5.0 pt%
        % \mbox{\hrefWithoutArrow{https://yourwebsite.com/}{yourwebsite.com}}%
        % \kern 5.0 pt%
        % \AND%
        \kern 5.0 pt%
        \mbox{\hrefWithoutArrow{https://linkedin.com/in/zachnorman02}{linkedin.com/in/zachnorman02}}%
        \kern 5.0 pt%
        \AND%
        \kern 5.0 pt%
        \mbox{\hrefWithoutArrow{https://github.com/zachnorman02}{github.com/zachnorman02}}%
    \end{header}

    \vspace{5 pt - 0.3 cm}

    \section{Experience}
    \begin{twocolentry}{
        May 2023 - Present
    }
    \textbf{Application Engineer}, Vanguard -- Malvern, PA \textit{fillin} \end{twocolentry}
    \vspace{0.10 cm}
    \begin{onecolentry}
      \begin{highlights}
      \end{highlights}
    \end{onecolentry}
    \vspace{0.2 cm}
    \begin{twocolentry}{
        Feb 2022 - Nov 2022
    }
    \textbf{Contracted Developer}, Orita \end{twocolentry}
    \textit{Python, Django, HTML, JavaScript, Svelte}
    \vspace{0.10 cm}
    \begin{onecolentry}
      \begin{highlights}
        \item Created a Python program to generate a PDF of a business analytics report given a JSON file.
        \item Implemented front-end design for user login, data upload, and e-commerce account connection for companies to easily access and upload their data for analysis
      \end{highlights}
    \end{onecolentry}
    \vspace{0.2 cm}
    \begin{twocolentry}{
        Jan 2022 - Sep 2022
    }
    \textbf{Developer Co-op}, Baltimore Orioles -- Baltimore, MD \end{twocolentry}
    \textit{Python, JavaScript, Django, React}
    \vspace{0.10 cm}
    \begin{onecolentry}
      \begin{highlights}
        \item Developed an amateur baseball player dashboard page containing data on draft-eligible players including implementing data retrieval and aggregation, and front-end interface with various filtering options.
        \item Wrote data processes and AWS Lambda functions to aggregate or correct data from games and update database tables
        \item Added edit features to game report audit pages, including the ability to add, delete, or reorder rows of data, and clear data columns, improving ability for coaches and analysts to correct and use game data.
      \end{highlights}
    \end{onecolentry}

    \section{Education}
    \begin{twocolentry}{
      Aug 2023 - Aug 2025
    }
    \textbf{Villanova University}, MS in Applied Statistics and Data Science \end{twocolentry}
    \vspace{0.10 cm}
    \begin{onecolentry}
      \begin{highlights}
        % \item GPA: 3.9/4.0 (\href{https://example.com}{a link to somewhere})
        % \item \textbf{Coursework:} Computer Architecture, Comparison of Learning Algorithms, Computational Theory
      \end{highlights}
    \end{onecolentry}
    \vspace{0.2 cm}
    \begin{twocolentry}{
        Sep 2020 - Apr 2023
    }
    \textbf{Northeastern University}, BS in Computer Science and Mathematics\end{twocolentry}
    \vspace{0.10 cm}
    \begin{onecolentry}
      \begin{highlights}
          % \item GPA: 3.9/4.0 (\href{https://example.com}{a link to somewhere})
          % \item \textbf{Coursework:} Computer Architecture, Comparison of Learning Algorithms, Computational Theory
          % \item \textbf{Extracurriculars:}
      \end{highlights}
    \end{onecolentry}
    
    \section{Projects}
        \begin{twocolentry}{
            \href{https://github.com/sinaatalay/rendercv}{github.com/name/repo}
        }
            \textbf{Multi-User Drawing Tool}\end{twocolentry}

        \vspace{0.10 cm}
        \begin{onecolentry}
            \begin{highlights}
                \item Developed an electronic classroom where multiple users can simultaneously view and draw on a "chalkboard" with each person's edits synchronized
                \item Tools Used: C++, MFC
            \end{highlights}
        \end{onecolentry}


        \vspace{0.2 cm}

        \begin{twocolentry}{
            \href{https://github.com/sinaatalay/rendercv}{github.com/name/repo}
        }
            \textbf{Synchronized Desktop Calendar}\end{twocolentry}

        \vspace{0.10 cm}
        \begin{onecolentry}
            \begin{highlights}
                \item Developed a desktop calendar with globally shared and synchronized calendars, allowing users to schedule meetings with other users
                \item Tools Used: C\#, .NET, SQL, XML
            \end{highlights}
        \end{onecolentry}


        \vspace{0.2 cm}

        \begin{twocolentry}{
            2002
        }
            \textbf{Custom Operating System}\end{twocolentry}

        \vspace{0.10 cm}
        \begin{onecolentry}
            \begin{highlights}
                \item Built a UNIX-style OS with a scheduler, file system, text editor, and calculator
                \item Tools Used: C
            \end{highlights}
        \end{onecolentry}

    \section{Technologies}
    \begin{onecolentry}
      \textbf{Languages:} Python, R, JavaScript/TypeScript, Java, SQL, HTML
    \end{onecolentry}
    \begin{onecolentry}
      \textbf{Frameworks/Libraries:} React, Svelte, Django, pandas, Plotly, scikit-learn, D3, tidyverse
    \end{onecolentry}
    \begin{onecolentry}
      \textbf{Tools:} Git, JetBrains IDEs, Visual Studio Code, Jupyter Notebook, AWS (Glue, Lambda, S3, IAM), Figma
    \end{onecolentry}
\end{document}